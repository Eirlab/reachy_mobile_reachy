%%%%%%%%%%%%%%%%%%
%  Introduction  %
%%%%%%%%%%%%%%%%%%
\setcounter{page}{1}
\section*{Introduction} \addcontentsline{toc}{section}{Introduction} \label{sec:introduction}

Depuis la révolution industrielle, les automates puis la robotique tendent à aider l'humain dans les différentes tâches qu'il doit réaliser, principalement dans les tâches de production en usine. Cependant, nombre de robots possèdent sont performants lorsqu'ils sont isolés (parfois en cage) et que l'humain n'est là que pour contrôler globalement l'ensemble des procédés. Imaginer un robot évoluant librement dans un magasin sans gêner, ni physiquement, ni psychologiquement, les usagers soulève des problématiques.

Depuis 1996, J. E. Colgate, W. Wannasuphoprasit et M. A. Peshkin ont proposé une définition de la notion de cobotique, cette notion a été subdivisée avec le temps en trois grands domaines de la cobotique. D'une part la cobotique industrielle permettant de répondre aux tâches difficiles et pénibles ou à très faible valeur ajoutée. Le cobot asiste en direct les gestes de l'opérateur en démultipliant ses capacités pour manipuler en sécurité des pièces chaudes, lourdes, encombrantes, petites ... La cobotique médicale, parfaitement illustrée par le cobot \href{https://fr.wikipedia.org/wiki/Da_Vinci_(chirurgie)}{Da Vinci} qui permet d'assister un chirurgien lors d'une opération. Finalement la cobotique qui va nous intéresser pendant ce projet, la cobotique dites conviviale, elle consiste à utiliser des robots, le plus souvent humanoïdes (par exemple le robot \href{https://fr.wikipedia.org/wiki/NAO_(robotique)}{NAO}), pour établir des communications et rendre des services. Cette utilisation prend son sens dans la robotique de service, où le robot humanoïde est par exemple appelé à guider, interagir et rendre des services\footnote{Exemple d'un robot de service à l'aéroport de Genève : \url{https://www.youtube.com/watch?v=Jdc_AmLVlVI}}.

C'est dans ce cadre de la cobotique de service que s'inscrit notre projet, en effet, notre client souhaite disposer d'un robot de service dans le cadre des 100 ans de l'ENSEIRB-MATMECA. L'objectif de ce robot est de fournir du contenu utile aux visiteurs comme leur indiquer la localisation des salles, des toilettes, proposer des jeux interactifs, etc. Pour se faire nous avons à notre disposition deux robots indépendants, d'une part un \href{https://ez-wheel.com/fr/kit-de-developpement-pour-agvamr}{robot de navigation EZ-WHEEL} et un \href{https://www.pollen-robotics.com/}{robot semi-humanoïde REACHY}. L'objectif de ce projet est d'intégrer les deux robots ensemble sur l'environnement ROS. Cette intégration couvre plusieurs domaines puisque nous devons à la fois réaliser l'intégration mécatronique, réaliser les algorithmes de navigation adéquats à ce nouveau robot et prévoir des interactions avec les visiteurs pour que ce robot soit une réelle valeur ajoutée à l'événement et non une curiosité à la limite de la vallée de l'étrange. Cela nous amène donc à la problématique générale de ce projet, comment concevoir un cobot destiné aux interactions humaines lors d'événements ?

% Comment concevoir un cobot destiné aux intéractions humaines lors d'évenements ?

Le contexte de ce projet soulève d'abord une problématique de foule, en effet le robot devra évoluer dans un environnement densément peuplé, il devra même faire plus qu'une simple navigation dans cet environnement, il devra être capable de détecter les humains et interagir avec eux de manière adéquate. Dans cet optique, nous allons orienter notre état de l'art selon trois axes, premièrement les questions de navigation dans un environnement dynamique et incertain sont traitées. Nous rajouterons ensuite le fait qu'un humain n'est pas un simple objet en mouvement, mais qu'il obéit à certaines règles sociales. Pour que notre robot soit suffisamment accepté il est nécessaire d'étudier ces règles sociales pour, \textit{in fine}, en faire un modèle et l'intégrer au robot. Finalement notre robot est amené à interagir avec les gens et à réaliser des jeux avec lui. Que ce soit pour l'interaction ou pour les jeux il est nécessaire qu'il possède la capacité de voir et de comprendre la sémantique derrière la vision, nous traitons donc des différents algorithmes qui seront utilisées.