%%%%%%%%%%%%%%%%%%
%  Conclusion  %
%%%%%%%%%%%%%%%%%%

\section*{Conclusion}
\addcontentsline{toc}{section}{Conclusion} \label{sec:conclusion}

Cet état de l'art avance différentes parties de la construction d'un cobot de service, plus particulièrement ce document s'attarde sur la navigation d'un robot, ses interactions avec les humains et le contenu de ces interactions. Tout d'abord en abordant les problématiques de modélisation de l'environnement en fonction de nos capteurs et en présentant un modèle rependu : la modélisation par grille d'occupation probabiliste. Une fois cette modélisation avancée il est démontré les différents moyens de planifier une trajectoire dans un environnement comme une foule. Les algorithmes délibératifs et plus particulièrement l'algorithme \textit{riskRRT} est présenté pour permettre de naviguer de manière la plus fiable possible dans un environnement dynamique et incertain.

Une fois que la navigation est mise en place, un deuxième aspect d'un cobot de service est abordé : les interractions qu'il doit avoir avec les humains. Ces interactions sont discutées selon deux axes, d'une part sur le comportement que doit adopter un cobot en présenter les règles sociales élémentaires et la théorie proxémique. D'autre part, la question de la communication entre un robot et une machine est explorée pour converger vers une conversation non verbale basée sur la détection de squelette. 

Finalement, toujours dans le but des interactions avec l'humain, il est discuté des techniques de mise en places des différents jeux en utilisant du \textit{machine learning}. Plus particulièrement les différentes techniques de \textit{machines learning} sont étudiées pour converger vers des techniques de \textit{deep learning} permettant de mettre en place ces jeux.

Ces trois axes d'études ne représentent qu'une partie des problèmes soulevés par la construction d'un cobot de service. Nous n'avons pas discuté, par exemple, de la taille que le cobot final doit avoir ou sa réaction attendue en fonction des différents problèmes qu'il va rencontrer. Comment réagir, par exemple, si un enfant tire sur le bras du robot fortement ?

En conclusion ce document définit et précise l'existant et les solutions qui existent pour résoudre des problèmes typiques à la construction d'un robot de service. Les solutions avancées dans ce document permettront de résoudre des problèmes majeurs dans la construction de notre robot de service, mais ne permettront pas à elles seule la construction d'un robot de service parfait.

 % à finir et reprendre
